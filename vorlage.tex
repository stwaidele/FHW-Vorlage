% Vorgaben Assignment aus Studienheft SQL03
% Formatvorgaben fuer den Text
% Umfang: 8 - 10 Seiten (inkl. Abbildungen und Tabellen, aber ohne Deckblatt, % Gliederung und Literaturverzeichnis, Eidesstattliche Erklaerung)
% Zeilenabstand: 1,5
% Schriftart: frei
% Schriftgrad: 12 pt
% Variablen, physikalische Groessen und Funktionszeichen werden kursiv gedruckt.
% Korrekturrand: links: 4,5 cm, rechts 2,0 cm, oben und unten jeweils 3,0 cm
% Deckblatt: (Adresse, AKAD-E-Mail-Adresse, Immatrikulationsnummer, Modul-
% bezeichnung, Thema, Datum, Felder für Korrektor)
% Gliederung (1 Seite)
% Literaturverzeichnis (3 - 5 Literaturquellen  z. B. Lehrbuecher, aktuelle Fachartikel recherchieren)
% Eidesstattliche Erklaerung (unterschrieben und fest eingebunden)
% Bearbeitungsdauer: 2 Monate


\documentclass[a4paper,12pt]{article}
\usepackage[ngerman]{babel}
\usepackage[authoryear]{natbib}
\usepackage{multibib}
\newcites{web}{Internetquellen}
\usepackage[nottoc]{tocbibind} % Anzeigen des Literaturverzeichnisses im TOC
\usepackage{epsfig}
\usepackage{supertabular}
\usepackage{wrapfig}
\usepackage{multirow}
\usepackage[onehalfspacing]{setspace}
\usepackage{listings}
\usepackage{mathptmx}
\usepackage{geometry}
%\usepackage{times}
%\usepackage{courier}
%\usepackage{lmodern} 
%\usepackage[scaled]{uarial}
\usepackage{helvet}
\renewcommand*{\familydefault}{\sfdefault}

\usepackage{setspace}
\usepackage{textcomp}
\usepackage[T1]{fontenc}
\usepackage[utf8]{inputenc}
\usepackage{fancyhdr}
\usepackage{float} % Notwendig fuer figure[h]
\usepackage[printonlyused]{acronym}
\usepackage{lipsum}
\usepackage{background}

\newif\iflistoffigures
\newif\iflistoftables
\newif\ifacronym
\newif\ifsperrvermerk
\newif\ifichweissdassichfuerdieeinhaltungdervorgabenselbstverantwortlichbin

%Titel
\newcommand*{\TitelMarkedup}{Behavioral Targeting \\ als Instrument des Online-Marketing} 
\newcommand*{\Titel}{Behavioral Targeting als Instrument des Online-Marketing} 

%Betreff
\newcommand*{\Betreff}{Hausarbeit im Studiengang ... \\ im Rahmen des Seminars ...} 

\newcommand*{\Semester}{Fachsemesterzahl} 


%Vor- und Nachname
\newcommand*{\Tags}{Behavioral Targeting, Marketing, Online, Online-Marketing}

%Betreuer
\newcommand*{\Betreuer}{Titel und vollständiger Name des Dozenten / der Dozentin} 

%Vor- und Nachname
\newcommand*{\Name}{Vorname Name des Autors bzw. der Autorin}

%Straße und Hausnummer
\newcommand*{\Strasse}{Anschrift} 

%Plz und Ort
\newcommand*{\PlzOrt}{der/des Studierenden} 

%Immatrikulationsnummer
\newcommand*{\Immatrikulationsnummer}{102 81 71}

%Email 
\newcommand*{\Email}{E–Mail Adresse} 

% Verzeichnisse (Wenn nicht benötigt, Zeile mit % auskommentieren oder löschen

%% Abbildungsverzeichnis 
%\listoffigurestrue
%% Tabellenverzeichnis
%\listoftablestrue
%% Abkürzungsverzeichnis
% \acronymtrue
%% Sperrvermerk
%\sperrvermerktrue

%\ichweissdassichfuerdieeinhaltungdervorgabenselbstverantwortlichbintrue

% Wittwen und Waisen verhindern
\clubpenalty10000
\widowpenalty10000
\displaywidowpenalty=10000

\usepackage[activate={true,nocompatibility},final,tracking=false,kerning=true,spacing=true,factor=1100,stretch=20,shrink=20]{microtype}
\usepackage[flushmargin,hang,ragged]{footmisc}

\renewcommand{\bflabel}[1]{\normalfont{\normalsize{#1}}\hfill}
\usepackage{fancyhdr}

\makeatother

\geometry{a4paper, left=40mm, right=20mm, top=30mm, bottom=30mm}

\pagenumbering{roman}
%% Definition for Codeschnipsel im Fließtext
\newcommand{\code}{\texttt}
% \newcommand{\buzz}{\textit}
\newcommand{\buzz}{\textit}

\newcommand{\todo}[1]{\fbox{\parbox{\textwidth}{\textbf{To do:} #1}}}
%\newcommand{\myref}[1]{„\ref{#1}~\nameref{#1}“}
\newcommand{\myref}[1]{\textit{\ref{#1}~\nameref{#1}}}

%% Für Codeblöcke mit Syntax-Highlighting
%% http://www.ctan.org/tex-archive/macros/latex/contrib/minted/
\usepackage{minted}
\definecolor{bg}{rgb}{0.95,0.95,0.95}

\backgroundsetup{%
  pages=some,
  firstpage=true,
  scale=1,       %% change accordingly
  angle=0,       %% change accordingly
  opacity=1,    %% change accordingly
  color =black,  %% change accordingly
  contents={\begin{tikzpicture}[remember picture,overlay]
        \node at (current page.center) {\includegraphics[width=21cm]{briefpapier}};    
    \end{tikzpicture}}
}


\usepackage[
	pdftitle={TODO Titel},
	pdfsubject={TODO Betreff},
	pdfauthor={TODO Name},
	pdfkeywords={TODO Taga},
	hyperfootnotes=false,
	colorlinks=true,
	linkcolor=black,
	urlcolor=black,
	citecolor=black
]{hyperref}


\begin{document}
\pagenumbering{arabic}

\parskip=1em
\parindent=0cm

\newgeometry{left=40mm, right=20mm, top=30mm, bottom=15mm}
%description: Deckblatt in Deutsch
%% Basierend auf einer TeXnicCenter-Vorlage von Tino Weinkauf.
%% sowie der akad-vorlage von Daniel Falkner
%%%%%%%%%%%%%%%%%%%%%%%%%%%%%%%%%%%%%%%%%%%%%%%%%%%%%%%%%%%%%%

%%%%%%%%%%%%%%%%%%%%%%%%%%%%%%%%%%%%%%%%%%%%%%%%%%%%%%%%%%%%%
%% Deckblatt
%%%%%%%%%%%%%%%%%%%%%%%%%%%%%%%%%%%%%%%%%%%%%%%%%%%%%%%%%%%%%
%%
%% ACHTUNG: Sie benötigen ein Hauptdokument, um diese Datei
%%          benutzen zu können. Verwenden Sie im Hauptdokument
%%          den Befehl "\input{dateiname}", um diese
%%          Datei einzubinden.
%%

\begin{titlepage}
%\thispagestyle{empty}

% \hspace*{-40mm}\includegraphics[width=21cm]{briefpapier.png}

\begin{spacing}{1.5}
\vspace*{2cm}
\begin{center}
\vfill
\vfill
\vfill
\vfill
\LARGE
\textbf{\TitelMarkedup}
\vfill
\vfill
\vfill
\vfill
\Large
\textbf{\Name} \\
\vfill
\vfill
\vfill
\normalsize
\Betreff
\vfill
bei
\vfill
\Betreuer
\end{center}
\vfill
\vfill
\vfill
\vfill
\vfill
\vfill
\vfill
\vfill	
\Strasse \\
\PlzOrt \hfill \Semester\\ 
\href{mailto:\Email}{\Email} \hfill \today
%%Datum der Abgabe - am besten selbst reinschreiben.
%\hfill \raisebox{3cm}{\colorbox{bg}{\hspace{5.78cm}\vspace*{10cm}}}
\vspace*{1.5cm}
\end{spacing}
\end{titlepage}
\backgroundsetup{contents={}}

\normalsize

\newgeometry{left=40mm, right=20mm, top=30mm, bottom=30mm}

\normalsize

\pagestyle{fancy}
\fancyhead{}
\fancyhead[LO,RE]{\textsc{\Titel}}
\fancyhead[RO,LE]{\thepage}
\fancyfoot[CO,CE]{}
\setlength{\headheight}{15pt}

\begin{spacing}{1.0} % Verzeichnisse werden mit einzeiligem Abstand gesetzt
\parskip=0em
\newpage

% Inhaltsverzeichnis
\setcounter{tocdepth}{3}
\renewcommand{\contentsname}{Gliederung}
\tableofcontents 
\newpage

% Tabellenverzeichnis
\iflistoftables
\listoftables 
\newpage
\fi

% Abbildungsverzeichnis
\iflistoffigures
\listoffigures 
\newpage
\fi

% Abkürzungsverzeichnis
\ifacronym
\section*{Abkürzungsverzeichnis}
\addcontentsline{toc}{section}{Abkürzungsverzeichnis} 
\begin{acronym}[\hspace{5cm}]
	\acro{CAPTCHA}{Completely Automated Public Tourint Test to Tell Computers and Humans Apart}
	\acro{CSS}{Cascading Style Sheets}
	\acro{DDoS}{Distributed Denial of Service}
	\acro{DoS}{Denial of Service}
	\acro{ERP}{Enterprise Resource Planning}
	\acro{HTML}{Hypertext Markup Language}
	\acro{HTTPS}{Hypertext Tranfsfer Protocol Secure}
	\acro{HTTP}{Hypertext Tranfsfer Protocol}
	\acro{IoT}{Internet of Things}
	\acro{OLAP}{Online Analytical Processing}
	\acro{OWASP}{Open Web Application Security Project}
	\acro{PDO}{PHP Data Objects}
	\acro{SCM}{Supply Chain Management}
	\acro{SQL}{Structured Query Language}
	\acro{WWW}{World Wide Web}
	\acro{XSS}{Cross Site Scripting}

%%
%	
%	\acro{}{}
%	\acro{}{}
%	\acro{}{}
%	\acro{}{}
%	\acro{}{}
%	\acro{}{}
%	\acro{}{}
%	\acro{}{}
	\acro{Test}{Wird nicht im Text verwendet und taucht auch nicht im Verzeichnis auf}
\end{acronym}

\fi

\parskip=1em
\end{spacing} 

\clearpage

% \nocite{*} 

\begin{spacing}{1.5} % Zeilenabstand: 1,5 fuer den Textteil

\section{Einleitung}
\label{sec:einleitung}

\subsection{Begründung der Problemstellung}
\label{sec:problemstellung}

\lipsum[1]\footnote{Auf die zusätzliche Angabe der weiblichen Form wird zugunsten der Lesbarkeit verzichtet.}

\subsection{Ziele dieser Arbeit}
\label{sec:ziele}

\textbf{Ziel dieser Arbeit ist es, \lipsum[2]}

Hierzu werden zunächst im Kapitel~\myref{sec:grundlagen} die für diese Arbeit relevanten Begriffe\footnote{vgl. \citeweb{gartneriot}, Abschnitt 3} und Konzepte\footnote{vgl. \cite{orwell}, Seite 7} definiert, bevor im Kapitel~\myref{sec:hauptteil}

\lipsum[3]

\subsection{Abgrenzung}
\label{sec:abgrenzung}

\lipsum[14]









\section{Grundlagen}
\label{sec:grundlagen}


\section{Marktanalyse}
\label{sec:markt}

\subsection{Angebotsanalyse}

\subsection{Nachfrageanalyse}

\subsubsection{Usecase: Stopover, privat, PKW}
\todo{mit oder ohne Haustier!}
\todo{Deutsche Rentner in Spanien}
\todo{Skifahrende Holländer}

\subsubsection{Usecase: Stopover, privat, Fahrrad}
\todo{Rheinradweg, Südschwarzwaldradweg}

\subsubsection{Usecase: Mehrere Tage, privat}
\todo{Radfahren, Markgräflerland, etc}


\subsubsection{Usecase: Stopover, dienstlich}
\todo{Außendienstler}

\subsubsection{Usecase: Mehrere Tage, dienstlich}
\todo{Montagemitarbeiter}

\subsection{Buchungskanäle}

\subsubsection{Direktbuchung}
\todo{HoherAnteil sowohl an Zufalls– als auch an Stammgästen}

\subsubsection{Internetportale}
\subsubsection{Ortsansässige Firmen}


\section{Umfeldanalyse}
\label{sec:umfeld}


\section{Unternehmensanalyse}
\label{sec:unternehmen}


\section{Unternehmensziele}
\label{sec:uziele}

\subsection{Erarbeitung}

\subsection{Realisierung}

\subsection{Überprüfung}

\section{Marketingziele}
\label{sec:mziele}

\subsection{Erarbeitung}

\subsection{Realisierung}

\subsection{Überprüfung}

\section{Marketingstrategien}
\label{sec:strategien}

\subsection{Erarbeitung}

\subsection{Realisierung}

\subsection{Überprüfung}

\section{Marketingmix}
\label{sec:mix}

\subsection{Erarbeitung}

\subsection{Realisierung}

\subsection{Überprüfung}

\section{Marketingcontrolling}
\label{sec:controlling}

\todo{Je nach dem, wie die Abschnitte „Überprüfung“ der Kapitel \myref{sec:ziele}, \myref{sec:strategien} und \myref{sec:mix} gestaltet werden, kann hier evt. nur eine Übersicht oder eine tabellarische Zusammenfassung nach Überprüfungszyklus stehen}
\section{Fazit \& Ausblick}
\label{sec:schluss}

\subsection{Fazit}
\label{sec:fazit}

\lipsum[6-8]

\subsection{Ausblick}
\label{sec:ausblick}

\lipsum[9-10]

\include{anhang}

\end{spacing}

\clearpage

%\pagestyle{plain}
\renewcommand{\refname}{Literatur-- und Quellenverzeichnis}
%\bibliographystyle{apalike}
%\bibliographystyle{alpha}
\bibliographystyle{dcu}
\bibliography{literatur}

\bibliographystyleweb{dcu}
\bibliographyweb{literatur}
\onehalfspacing
\clearpage

%\pagestyle{empty} 
%\thispagestyle{empty}

\vspace*{\fill}
\vfill
\vfill
\begin{center}
{\Large Eidesstattliche Erklärung}
\end{center}
\vfill
\noindent
Ich versichere, dass ich die beiliegende Abschlussarbeit selbstständig verfasst, keine anderen als die angegebenen Quellen und Hilfsmittel benutzt sowie alle wörtlich oder sinngemäß übernommenen Stellen in der Arbeit gekennzeichnet habe. 
\vfill
\vfill
\vfill
\rule[0.5ex]{6.5cm}{1pt}
\hspace{1.3cm}
\rule[0.5ex]{6.5cm}{1pt}
\\(Datum, Ort)
\hspace{6.3cm}
(Unterschrift)
\vfill
\vfill
\vfill
\ifsperrvermerk
\begin{center}
{\Large Sperrvermerk}
\end{center}
\vfill
Diese Arbeit enthält vertrauliche Daten und Informationen des betreuenden
Unternehmens. Sie darf Dritten deshalb nicht zugänglich gemacht werden.

Die drei für die Prüfung notwendigen Exemplare verbleiben beim Prüfungsamt und bei
den beiden betreuenden Hochschullehrern.
\fi
\vfill
\vfill
\vfill
\clearpage

\ifichweissdassichfuerdieeinhaltungdervorgabenselbstverantwortlichbin
\else
\section*{Gebrauchsanweisung}
\label{sec:gebrauchsanweisung}
\subsection*{Rechtliche Informationen}
Diese \LaTeX--Vorlage wurde von Stefan Waidele\footnote{\url{mailto:Stefan@Waidele.info} bzw. \url{http://Stefan.Waidele.info}} erstellt. Als Basis diente die Vorlage für AKAD--Assignments\footnote{\url{https://github.com/derdanu/akad-vorlage}} von Daniel Falkner.

Diese Vorlage wurde unabhängig von der FH--Westküste (FHW) entwickelt. Die Vorgaben der FHW für wissenschaftliche Arbeiten sind unter\\ \url{http://www.fh-westkueste.de/studierende/abschlussarbeiten/} zu finden. 

\textbf{Jeder Nutzer dieser Vorlage ist selbst für deren Einhaltung verantwortlich.}

Diese Vorlage darf unter den Bedingungen der Creative Commons Lizenz „CC BY–SA 4.0“ verwendet werden. Weitere Informationen hierzu sind unter \url{http://creativecommons.org/licenses/by-sa/4.0/deed.de} verfügbar.

Die Rechte an den mit dieser Vorlage erstellten wissenschaftlichen Arbeiten oder sonstigen Werken verbleiben selbstverständlich beim jeweiligen Autor und fallen \textbf{nicht} unter die Restriktionen der oben genannten CC–Lizenz.

Leider ist es auch rechtlichen Gründen nicht möglich, das offizielle Logo der FHW zusammen mit dieser Vorlage weiterzugeben. Mit einem Grafikprogrogramm wie z.B. GIMP kann dies jedoch individuell eingefügt werden.

\subsection*{Sonstiges}

Um das Kapitel~\myref{sec:gebrauchsanweisung} zu entfernen, ist in die Datei \\
\code{einstellungen.tex} die folgende Zeile einzufügen:

\code{\textbackslash{}ichweissdassichfuerdieeinhaltungdervorgabenselbstverantwortlichbintrue}

\fi
\end{document}

