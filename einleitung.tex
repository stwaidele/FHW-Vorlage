\section{Einleitung}
\label{sec:einleitung}

\subsection{Begründung der Problemstellung}
\label{sec:problemstellung}


\subsection{Ziele dieser Arbeit}
\label{sec:zieledieserarbeit}

\textbf{Ziel dieser Arbeit ist es, eine vollständige Marketingkonzeption für das Hotel--Restaurant Krone in Neuenburg am Rhein zu erstellen. Bereits vorhandene Elemente sind hierbei zu beschreiben, für die Zukunft fortzuschreiben und durch die fehlenden Elemente zu ergänzen. }

Hierzu werden zunächst im Kapitel~\myref{sec:grundlagen} die für diese Arbeit relevanten Begriffe und Konzepte definiert, bevor im Kapitel~\myref{sec:ziele} die Marketingziele des Unternehmens beschrieben bzw. neu bestimmt werden. Das Kapitel~\myref{sec:strategien} entwickelt basierend auf ebendiesen Strategien, welche dann im Kapitel~\myref{sec:mix} zu einem konkreten Marketingmix weitergeführt werden.

Hierbei werden auf jeder dieser Ebenen auch Messgrößen für die Zielerreichung definiert, die der Erfolgskontrolle im Rahmen des Marketingcontrollings zu Grunde gelegt werden können.

\subsection{Abgrenzung}
\label{sec:abgrenzung}









